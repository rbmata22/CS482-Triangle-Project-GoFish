\clearpage
\section{Reflection}

\subsection{Team Reflection}
The project gave us numerous lessons, significant barriers, and key facilitators that shaped our team’s experience and outcomes. One of the primary lessons learned was the importance of communication and accountability within a team. Lots to reflect on with how starting tasks earlier, collaborating more frequently, and ensuring equal understanding of new technologies could have improved the overall workflow. Additionally, planning and thorough preparation were highlighted as very important. Particularly in overcoming challenges with  unfamiliar tools like Firebase and Jest. The importance of allocating time for team members to teach and support one another was another valuable takeaway, as some members struggled initially with steep learning curves and minimal guidance. The barriers encountered throughout the project included unfamiliarity with certain frameworks and technologies which introduced significant complexity. Jest testing also posed challenges and more knowledge than the team initially has was needed. Individual time management and procrastination were additional hurdles which would lead to added stress and inefficiency. These barriers proved the need for early preparation and task delegation. Despite the barriers, several facilitators contributed to the team’s progress and ultimate success. AI helped with debugging, frameworking, and understanding new technologies. It provided insights into resolving complex problems and saved time that would have otherwise been spent searching for solutions manually. Team collaboration also played a big role near the end of the project. Members then helped guide each other with challenges and ensuring progress remained steady.  Overall, improved teamwork and the use of AI allowed the team to overcome barriers and deliver a successful project.

\subsection{Marley's Reflection}
If I were to start the project today, first and foremost, I would make sure that everyone involved had an equal or close to equal understanding of all the new technologies used and all aspects of the project. The beginning would involve a lot of planning and understanding. I would also plan accordingly to account for varying levels of understanding among team members. This means that if a team member didn’t understand something, the knowledgeable team member could spend time teaching while ensuring that the less knowledgeable team member still had tasks to work on. Additionally, I would make more team collaboration time mandatory, where we all code together, so our overall teamwork could improve. I believe that group members should be held more accountable for doing things last minute. Overall, I think most of what we did was good, but it often felt like everyone was doing their own thing. If someone lagged behind or was confused, they were mostly left to figure it out on their own, especially at the beginning. Near the end, it did get a little better, but at least on my end, I had to take an entire sprint to learn all the new things with minimal help.



\subsection{Silas's Reflection}
If I were to start the project today, i would’ve made sure to prioritize communication and find a accurate way to measure workload between members. A barrier we had was understanding the Jest framework as it was unpredictable at times to work with. When testing would fail it was difficult to understand why or how to fix it due to our lacking knowledge on the new technology. I would say me and Ryland were main facilitators due to our existing knowledge of React and Firebase. We were able to help the other teammates out if they were confused. A majority of our team was organized and communicated their work. I tried to be a leader for our team by being consistent with my work, actively communication and updating teammates on progress of both my work, and the state of the current project. I learned a lot about communication, task allocation and workload management. I also learned how to create deliverables for a client through 3 sprints.

\subsection{Ryland's Reflection}
Had we started the projected today I definitely would have utilized my time much more efficiently. Whether it be starting tasks much earlier or getting ahead on setting up and preparing for the next sprint. The main barriers that had been faced was with Firebase as it was something new and unfamiliar, thus, there was a very steep learning curve especially with the large scale of this project. The main facilitators were each other as we all were experiencing obstacles that needed to be dealt with. Having one another to help with anything that goes wrong was very beneficial towards the project. The team was fairly organized, we kept our GitHub repository organized in terms of separating documentation and actual code. We were also efficient in assigning different stories or tasks and dividing up the work. Our leader would most likely be Silas as he tackled a large majority of the more difficult parts of this project. Not only this, but he was definitely the most active group member in terms of writing code, communication with the group, and overall being on top of things. After this project, I learned that software engineering requires a significant amount of concentration and organization. While the coding can be difficult, if a project is not organized and structured properly from the beginning, it will ultimately lead to a disaster over time. Thus, taking each stop of the process carefully and thoroughly checking that every part is completed properly is crucial to success.

\subsection{Chase's Reflection}
If the project had started today, I would have gotten an earlier jumpstart on it rather than procrastinating, and I agree with Ryland that there was a very tough learning curve involved in the software. The team was pretty organized, though I was slacking in the beginning. I’d agree that Silas was our team leader and kept us on track the best. I learned during this project that organization and communication is key when developing software

\newpage
\section{AI Impact}

\subsection{Marley's Experience}
The usage of AI definitely helped me. For example, it was very useful for frameworking and explaining how things work, saving me from having to search through textbooks for answers. Additionally, AI was extremely helpful during debugging. When I encountered issues in my code, I used AI to better understand error messages and identify potential fixes. This saved me time and allowed me to focus on improving my code rather than getting stuck on small issues. However, I made sure not to rely on AI too heavily. Instead of depending on it for complete solutions, I used it as a  tool to clarify concepts and guide my understanding. I still took the time to research and learn independently. Making sure I fully grasped the material and debugging process. This approach allowed me to balance AI's efficiency and developing my own problem solving and coding skills.

\subsection{Silas's Experience}
The usage of AI significantly sped up the process of doing things I was already familiar with doing. I would say it worked best once you KNOW what you’re looking for, like including specific libraries, icons, or tools. For example, I was familiar with Firebase’s Google Sign Up feature, so it was easy to integrate that feature into code when I make that an explicit requirement to include in the LLM’s output. UI drawups also helped with our Apps menu layout as well. LLM’s can read images and adjust CSS to match what the picture looks like. I learned about this over the summer when I found the website “jit.dev” which creates CSS for you based on the prompt you give. Dr. Opitz also showed us a really good example in class with Pac-Man, explaining that the precision of your prompts are key to good generated code. I used the logic there for the newer models like Claude, and the results were very good. I’ve learned that the more precise your prompt is, the more detailed the results were

\subsection{Ryland's Experience}
Overall, the usage of AI helped significantly with understanding the more complex tasks of this project. There were a few stories that required this assistance due to it being both difficult to understand and excessively time consuming had I tried to start from scratch. It proved useful in offering tips when I was dealing with very specific and unique problems, especially with Firebase. However, I mainly utilized it as a tool to build a beginning foundation/outline of the code I wanted to produce. It helped significantly as it had allowed me to have a base to build off of. From this, I would then be able to continue coding on my own for whatever task I was trying to accomplish, thus allowing for proper integration into the project.

\subsection{Chase's Experience}
The use of ai helped to break down the project into manageable pieces and develop some of the really tough code. It was helpful when I really didn’t know what to do next and gave a lot of tips. It was very helpful in understanding new technology that I had not worked with before, as well as in debugging code